\documentclass[12pt,a4paper]{report}
\usepackage{cite}
\usepackage{amsmath,amssymb,amsfonts}
\usepackage{amsthm}
\usepackage{algorithmic}
\usepackage{graphicx}
\usepackage{url}
\usepackage{caption}
\usepackage{float}
\usepackage{hyperref}\usepackage{adjustbox}
\usepackage{array}

\usepackage[left=2.5cm,right=2.5cm,top=2.5cm,bottom=2.5cm]{geometry}

\def\BibTeX{{\rm B\kern-.05em{\sc i\kern-.025em b}\kern-.08em
    T\kern-.1667em\lower.7ex\hbox{E}\kern-.125emX}}
\begin{document}

\title{Proposal\\
}


\pagenumbering{roman}
\setcounter{page}{1}
\thispagestyle{empty} 
\begin{titlepage}

    \begin{center}
        \vspace*{0.7cm}
         \Huge
        \textbf{PROPOSAL}
  
        \vspace{2cm}
          \Large
          Edwin H. Ng
                
        \vspace{2cm}
        
        edwin.ng@mail.mcgill.ca
                
        \vspace{2cm}

        \Large
        ID: 26976581\\
           \vspace{10cm}
        October 18, 2019\\
        \vspace{2cm}    
    \end{center}
\end{titlepage}


\chapter*{Brain Storming}
\begin{enumerate}
\item \textbf{Text Summarization}
\item Generating a condensed version of a passage while preserving its meaning is known as text summarization. Tackling this task is an important step towards natural language understanding. Summarization systems can be broadly classified into two categories. \textit{Extractive models} generate summaries by cropping important segments from the original text and putting them together to form a coherent sum- mary. \textit{Abstractive models} generate summaries from scratch without being constrained to reuse phrases from the original text.
\item Rush et al.: First to use neural
\item Given a sentence: make summarization
\item Sumit et al. : RNN
\item \textbf{STOCK PREDICTION BASED ON Text news}
\item Kaggle: already data
\end{enumerate}
\end{document}