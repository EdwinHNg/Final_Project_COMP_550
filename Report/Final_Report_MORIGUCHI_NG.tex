\documentclass[conference]{IEEEtran}
\IEEEoverridecommandlockouts
% The preceding line is only needed to identify funding in the first footnote. If that is unneeded, please comment it out.
\usepackage{cite}
\usepackage{amsmath,amssymb,amsfonts}
\usepackage{algorithmic}
\usepackage{graphicx}
\usepackage{url}
\usepackage{caption}
\usepackage{float}
\usepackage{hyperref}\usepackage{adjustbox}
\usepackage{array}

\newcolumntype{R}[2]{%
    >{\adjustbox{angle=#1,lap=\width-(#2)}\bgroup}%
    l%
    <{\egroup}%
}
\newcommand*\rot{\multicolumn{1}{R{90}{-1em}}}% no optional argument here, please!

\def\BibTeX{{\rm B\kern-.05em{\sc i\kern-.025em b}\kern-.08em
    T\kern-.1667em\lower.7ex\hbox{E}\kern-.125emX}}
\begin{document}

\title{Predicting Bitcoin Price using LSTM and Twitter Sentiment Analysis\\
%{\footnotesize \textsuperscript{*}Note: Sub-titles are not captured in Xplore and
%should not be used}
%\thanks{Identify applicable funding agency here. If none, delete this.}
}

\author{\IEEEauthorblockN{}
\IEEEauthorblockA{\textit{Haruki Moriguchi} \\
\textit{haruki.moriguchi@mail.mcgill.ca} \\
ID: 260665818}
\and
\IEEEauthorblockN{}
\IEEEauthorblockA{\textit{Edwin H.Ng} \\
\textit{edwin.ng@mail.mcgill.ca} \\
ID: 260732345}
}

\twocolumn[
  \begin{@twocolumnfalse}
    \maketitle
    \begin{center}
    \begin{minipage}{0.5\linewidth}
    \begin{abstract}
	\textbf{We consider the problem of predicting annual income using Machine Learning tools. The Census Income dataset has been examined for our purpose by using Random Forest Algorithm. The dataset is available at the "​UC Irvine Machine Learning Repository"\footnote{\url{http://archive.ics.uci.edu/ml/datasets/Adult}}. We also tried to use Generalized linear model (GLM), as a benchmark for testing our proposed algorithm.}
\end{abstract}
\begin{IEEEkeywords}
\begin{center}
\textbf{Bitcoin, LSTM, Sentiment Analysis}
\end{center}
\end{IEEEkeywords}
    \end{minipage}
    \end{center}
  \end{@twocolumnfalse}
]


\section{Introduction}
	Predicting whether or not an individual attains a fixed level of income is of great practical significance, particularly in the pension plan sector in view of actuarial valuations.\\
	The primary contributions of this paper is to classify income levels using tree-based methods as follows: conditional inference tree and random forest. In order to compare their predictive power, we also perform classification based on simpler model, the logistic regression.  \\ 
	We proceed as follows. First, a short description of the dataset is performed. Section 3 lays out the methodology of our paper. Section 4 presents our results and the analysis of our different models. Finally we conclude in Section 5.     

 % ------------------------


\begin{thebibliography}{2}
\bibitem{b1} \url{https://en.wikipedia.org/wiki/Out-of-bag_error}
\bibitem{b2} \url{https://cran.r-project.org/web/packages/partykit/vignettes/ctree.pdf}
\end{thebibliography}

\end{document}
